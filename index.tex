% Options for packages loaded elsewhere
\PassOptionsToPackage{unicode}{hyperref}
\PassOptionsToPackage{hyphens}{url}
\PassOptionsToPackage{dvipsnames,svgnames,x11names}{xcolor}
%
\documentclass[
  a5paperpaper,
  DIV=11,
  numbers=noendperiod]{scrreprt}

\usepackage{amsmath,amssymb}
\usepackage{iftex}
\ifPDFTeX
  \usepackage[T1]{fontenc}
  \usepackage[utf8]{inputenc}
  \usepackage{textcomp} % provide euro and other symbols
\else % if luatex or xetex
  \usepackage{unicode-math}
  \defaultfontfeatures{Scale=MatchLowercase}
  \defaultfontfeatures[\rmfamily]{Ligatures=TeX,Scale=1}
\fi
\usepackage[]{libertinus}
\ifPDFTeX\else  
    % xetex/luatex font selection
\fi
% Use upquote if available, for straight quotes in verbatim environments
\IfFileExists{upquote.sty}{\usepackage{upquote}}{}
\IfFileExists{microtype.sty}{% use microtype if available
  \usepackage[]{microtype}
  \UseMicrotypeSet[protrusion]{basicmath} % disable protrusion for tt fonts
}{}
\makeatletter
\@ifundefined{KOMAClassName}{% if non-KOMA class
  \IfFileExists{parskip.sty}{%
    \usepackage{parskip}
  }{% else
    \setlength{\parindent}{0pt}
    \setlength{\parskip}{6pt plus 2pt minus 1pt}}
}{% if KOMA class
  \KOMAoptions{parskip=half}}
\makeatother
\usepackage{xcolor}
\setlength{\emergencystretch}{3em} % prevent overfull lines
\setcounter{secnumdepth}{5}
% Make \paragraph and \subparagraph free-standing
\makeatletter
\ifx\paragraph\undefined\else
  \let\oldparagraph\paragraph
  \renewcommand{\paragraph}{
    \@ifstar
      \xxxParagraphStar
      \xxxParagraphNoStar
  }
  \newcommand{\xxxParagraphStar}[1]{\oldparagraph*{#1}\mbox{}}
  \newcommand{\xxxParagraphNoStar}[1]{\oldparagraph{#1}\mbox{}}
\fi
\ifx\subparagraph\undefined\else
  \let\oldsubparagraph\subparagraph
  \renewcommand{\subparagraph}{
    \@ifstar
      \xxxSubParagraphStar
      \xxxSubParagraphNoStar
  }
  \newcommand{\xxxSubParagraphStar}[1]{\oldsubparagraph*{#1}\mbox{}}
  \newcommand{\xxxSubParagraphNoStar}[1]{\oldsubparagraph{#1}\mbox{}}
\fi
\makeatother


\providecommand{\tightlist}{%
  \setlength{\itemsep}{0pt}\setlength{\parskip}{0pt}}\usepackage{longtable,booktabs,array}
\usepackage{calc} % for calculating minipage widths
% Correct order of tables after \paragraph or \subparagraph
\usepackage{etoolbox}
\makeatletter
\patchcmd\longtable{\par}{\if@noskipsec\mbox{}\fi\par}{}{}
\makeatother
% Allow footnotes in longtable head/foot
\IfFileExists{footnotehyper.sty}{\usepackage{footnotehyper}}{\usepackage{footnote}}
\makesavenoteenv{longtable}
\usepackage{graphicx}
\makeatletter
\def\maxwidth{\ifdim\Gin@nat@width>\linewidth\linewidth\else\Gin@nat@width\fi}
\def\maxheight{\ifdim\Gin@nat@height>\textheight\textheight\else\Gin@nat@height\fi}
\makeatother
% Scale images if necessary, so that they will not overflow the page
% margins by default, and it is still possible to overwrite the defaults
% using explicit options in \includegraphics[width, height, ...]{}
\setkeys{Gin}{width=\maxwidth,height=\maxheight,keepaspectratio}
% Set default figure placement to htbp
\makeatletter
\def\fps@figure{htbp}
\makeatother

\KOMAoption{captions}{tableheading}
\makeatletter
\@ifpackageloaded{bookmark}{}{\usepackage{bookmark}}
\makeatother
\makeatletter
\@ifpackageloaded{caption}{}{\usepackage{caption}}
\AtBeginDocument{%
\ifdefined\contentsname
  \renewcommand*\contentsname{Содержание}
\else
  \newcommand\contentsname{Содержание}
\fi
\ifdefined\listfigurename
  \renewcommand*\listfigurename{Список Иллюстраций}
\else
  \newcommand\listfigurename{Список Иллюстраций}
\fi
\ifdefined\listtablename
  \renewcommand*\listtablename{Список Таблиц}
\else
  \newcommand\listtablename{Список Таблиц}
\fi
\ifdefined\figurename
  \renewcommand*\figurename{Рисунок}
\else
  \newcommand\figurename{Рисунок}
\fi
\ifdefined\tablename
  \renewcommand*\tablename{Таблица}
\else
  \newcommand\tablename{Таблица}
\fi
}
\@ifpackageloaded{float}{}{\usepackage{float}}
\floatstyle{ruled}
\@ifundefined{c@chapter}{\newfloat{codelisting}{h}{lop}}{\newfloat{codelisting}{h}{lop}[chapter]}
\floatname{codelisting}{Список}
\newcommand*\listoflistings{\listof{codelisting}{Список Каталогов}}
\makeatother
\makeatletter
\makeatother
\makeatletter
\@ifpackageloaded{caption}{}{\usepackage{caption}}
\@ifpackageloaded{subcaption}{}{\usepackage{subcaption}}
\makeatother

\ifLuaTeX
  \usepackage{selnolig}  % disable illegal ligatures
\fi
\usepackage{bookmark}

\IfFileExists{xurl.sty}{\usepackage{xurl}}{} % add URL line breaks if available
\urlstyle{same} % disable monospaced font for URLs
\hypersetup{
  pdftitle={Голос травмированного поколения},
  pdfauthor={Александра Булгакова},
  colorlinks=true,
  linkcolor={blue},
  filecolor={Maroon},
  citecolor={Blue},
  urlcolor={Blue},
  pdfcreator={LaTeX via pandoc}}


\title{Голос травмированного поколения}
\usepackage{etoolbox}
\makeatletter
\providecommand{\subtitle}[1]{% add subtitle to \maketitle
  \apptocmd{\@title}{\par {\large #1 \par}}{}{}
}
\makeatother
\subtitle{2012-2024}
\author{Александра Булгакова}
\date{2024-06-10}

\begin{document}
\maketitle

\renewcommand*\contentsname{Содержание}
{
\hypersetup{linkcolor=}
\setcounter{tocdepth}{2}
\tableofcontents
}

\bookmarksetup{startatroot}

\chapter*{Интро}\label{ux438ux43dux442ux440ux43e}
\addcontentsline{toc}{chapter}{Интро}

\markboth{Интро}{Интро}

\section*{2012-2024}\label{section}
\addcontentsline{toc}{section}{2012-2024}

\markright{2012-2024}

\href{./2012.qmd}{2012}

\href{./2017.qmd}{2017}

\href{./2020.qmd}{2020}

\href{./2022.qmd}{2022}

\href{./2023.qmd}{2023}

\href{./2024.qmd}{2024}

\bookmarksetup{startatroot}

\chapter{2012}\label{section-1}

\section{}\label{section-2}

Я хочу умереть, глядя в желтые листья.

Я хочу умереть, кутаясь в песочный плед.

Я хочу умереть, испачкав в краске кисти.

Я хочу умереть, оставляя на снегу след.

Но я умер вчера, до того как мне стукнуло двадцать,

Допивая горячий непентес и путаясь в дым,

Средь прочитанных книг, недописанных писем,

Где всегда оставался и буду один.

И я буду хотеть умереть пока буду дышать,

( и влюбляться поспешно)

Вновь и вновь чувствовать что-то внутри.

Для меня делать умереть --- уж давно неизбежно,

Сколько будут гореть средь души пустыри.

\begin{quote}
\emph{осень 2012}
\end{quote}

\section{}\label{section-3}

Мы вкалываем чувства,

--- теряя вечность,

Впитываем боль, находя, что падение

--- лишь путь в бесконечность.

Стираем клавишами строки

--- звездного неба.

Все так же слепо ищем истоки

--- счастья, но не находим,

И лишь тихо уходим.

С каждым новым вдохом

мы угасаем

Как недокуренные свечи, забывая,

Что мир сегодня --- вечен.

\begin{quote}
\emph{лето 2012}
\end{quote}

\section{}\label{section-4}

Мы сидели под бархатной луной,

Выкуривая воспоминания.

Пламя протягивало к нам свои сигаретные язычки.

Ты улыбался мне улыбкой,

В которой утонул бы сам океан.

И все было так же,

Как в серебристых снах,

Только немножечко волшебней.

\begin{quote}
\emph{лето 2012}
\end{quote}

\section{}\label{section-5}

Осенними днями лета,

Сладкими минутами головокружения,

В дыму одиночества

Равносильно бессоннице осознание,

Что ты тот, ради кого хочется умереть.

Шуршание сигареты во время вдоха

Может сказать о нас больше,

Чем все пустые люди вокруг.

И не важно, сколько голосов звучало прежде.

Теперь ощутим лишь задорный смех нашей дружбы.

\begin{quote}
\emph{лето 2012}
\end{quote}

\bookmarksetup{startatroot}

\chapter{2017}\label{section-6}

\section{ГОЛОС ТРАВМИРОВАННОГО
ПОКОЛЕНИЯ}\label{ux433ux43eux43bux43eux441-ux442ux440ux430ux432ux43cux438ux440ux43eux432ux430ux43dux43dux43eux433ux43e-ux43fux43eux43aux43eux43bux435ux43dux438ux44f}

Тусклые кольца табачного дыма

Тянутся прочь.

Где бы ты ни был,

Окутанный в ночь,

Измучен до пота,

Уставший,

Пропитанный горестной рвотой,

Иссякший,

В объятья забвенья упавший,

Изрядно подвыпивший, но все таки ставший,

Голосом

Травмированного поколенья,

Что ганжей да хересом глушит

Сердцебиение

Той мелочной бляди,

Зовущейся музой,

Нещадно штурмующей

Шаткие шлюзы

И без того разбитого подсознанья,

Надежды дарующей,

Чарующей

Вновь,

В конечном итоге

Придумавшей

Любовь

--- эту страшную штуку.

Вытяни руку

И сосчитай шрамы,

Порезы, окурков ожоги,

Сердечные раны.

Подумай о них как о свидетелях

Сгоревших мгновений,

Весны дуновений

Жизней прошедших,

Терзаний исполненных,

Но не ушедших

Бесследно.

Подумай о них

Как о тягостной дрёме,

О коме,

С которой ты вышел.

Теперь ты хоть знаешь о взломе

Собственной крыши.

Прими их как штампы того,

Что ты выжил.

Ведь где бы ты ни был,

Курящий,

Окутанный в ночь,

Болящий,

Шлющий поэзию прочь,

Иссякший,

Трясущийся хаотично,

В итоге ты жив,

Пускай и частично.

\begin{quote}
\emph{март 2017}
\end{quote}

\section{СЕРОЕ
НЕБО}\label{ux441ux435ux440ux43eux435-ux43dux435ux431ux43e}

В моем городе серое небо

И такие же серые птицы.

Оно видится где бы ты ни был,

Отражаясь в тумане сквозь лица

Мимолетно взглянувших прохожих

В глубину моего подсознанья,

Что уверены будто негоже

Быть в унынии столь юным созданиям.

Тем не менее, немое отчаянье

Никому не расскажет того,

Что нет узника в мире печальнее,

Чем поэт,

Что заперт

Внутри себя

Самого.

\begin{quote}
\emph{весна 2017}
\end{quote}

\section{}\label{section-7}

Деревьев колкими ветвями

Перечеркнул мое безоблачное небо

Очередной поэт души печальной,

Обиженный на происки Вселенной.

Ведь очень может быть, что даже Бродский

В душе моей набедокурил от и до:

Мотив всех горестей грядет, увы, не плотский,

Иначе б сердцу уже было все равно.

А сны мои --- гораздо ярче жизни ---

Переплетаясь с памятью Ведут

Куда-то глубоко, где даже извне

Крымских закатов теплотой сквозит уют.

Мне с тишиною в темноте приходит море,

И волнами уводит за собой,

Будто в мое немыслимое горе

С прибоем принесет такой простой Покой.

Бредя по улицам до крайности пустынным,

Себя напоминаю только тень,

Чьи раны еще близко не остыли

И вовсе не готовы встретить новый день.

Вы говорите: «Умирать не ново!» только

Пожалуй, так бы всем стало жить проще.

Однако, новый день уже не не задворках,

И в мире вновь одним больным поэтом больше.

Сверкнули фары в предрассветной мгле,

И вопль тормозов пронзил могильный век.

Водитель выскочил, о смерти распинаясь мне,

А я стою и думаю: ни фонарей здесь, ни аптек.

Одна лишь ночь, которою я слышу,

Что пешеходный к черту замело,

Что с алкоголем людям явно сносит крышу\ldots{}

--- Увы, сама пьяна я только болью,

Да на снега мне как-то все равно.

И в ответ ухмыльнусь крайне мрачно,

Пускай брани давно не боюсь:

На тот свет не спешу однозначно,

Но и на этом задержаться не стремлюсь.

\begin{quote}
\emph{февраль 2017}
\end{quote}

\bookmarksetup{startatroot}

\chapter*{2020}\label{section-8}
\addcontentsline{toc}{chapter}{2020}

\markboth{2020}{2020}

\section*{}\label{section-9}
\addcontentsline{toc}{section}{}

\markright{}

А помнишь солнце?

Когда у нас ещё было на него время.

Робкие рассветы, скрывающие в себе ночное бремя,

Пылкие закаты, большинство из которых не обошлось без виски.

Ты помнишь зарю, пришедшую к нам на вписку?

Морской воздух, каким он бывает лишь по утрам,

Безлюдную набережную,

Струящиеся бархатом волны и то,

С какой немыслимой силой они разбиваются о камень,

Превращаясь в миллиарды сияющих осколков.

Времени прошло сколько?

Мгновение и их подхватит ветер южный,

И унесёт куда-то за прибой.

Ты помнишь солнце, дорогой?

Кому теперь все это нужно?

\begin{quote}
\emph{февраль 2020}
\end{quote}

\bookmarksetup{startatroot}

\chapter*{2022}\label{section-10}
\addcontentsline{toc}{chapter}{2022}

\markboth{2022}{2022}

\section*{}\label{section-11}
\addcontentsline{toc}{section}{}

\markright{}

Зданий кирпичных фасадами

Тянутся воспоминания,

А я брожу по ним

Не в силах передать того,

Что словно отзвук безысходности

Закрыло меня в собственной душе.

\begin{quote}
\emph{лето 2022}
\end{quote}

\section*{}\label{section-12}
\addcontentsline{toc}{section}{}

\markright{}

Мы стояли у древнего дуба, окаймляющего Киев.

Мир казался таким безутешным

И в то же время полным причин продолжать свой путь.

За спиной красовались огни кем-то наполненных зданий.

У нас не было знаний (о них).

Лишь горстка любви и щепотка безмятежности,

Которую я утратила,

Когда выплакала все слезы,

Поливая чужие гортензии.

\begin{quote}
\emph{лето 2022}
\end{quote}

\section*{}\label{section-13}
\addcontentsline{toc}{section}{}

\markright{}

Если ты я могла выстрелить

В брюхо каждого, кто сжег мосты за спинами моих близких,

Я все равно предпочла бы заткнуть дуло цветком.

Ведь мир --- не математика,

И минус на минус

Никогда не даст плюс.

\begin{quote}
\emph{лето 2022}
\end{quote}

\section*{АКВАРИУМ}\label{ux430ux43aux432ux430ux440ux438ux443ux43c}
\addcontentsline{toc}{section}{АКВАРИУМ}

\markright{АКВАРИУМ}

Мою судьбу отняли так жестоко

Едва ль исполнилось четверть века.

Я вышел в подъезд одинокий

В попытках отыскать в себе человека.

Встал на цыпочки, заглянул в окно,

А за ним лишь утро и улица.

Выкурил сигаретку, и уже все равно,

Что мечты и надежды не сбудутся.

За окном суматошно и сумрачно все

И люди как рыбки в аквариуме:

Несутся куда-то, ударяясь о дно,

Ну мы-то знаем, плавали.

Под окном один автомобиль

--- и тот чужой.

Смотрю на него, и сам не свой.

Мол, в мире бессмысленно все, даже Солнце.

Закрой окошко и успокойся.

Одному прощаться с жизнью нелегко

Да как-то тесно,

Когда в собственной душе нет тебе места.

\begin{quote}
\emph{осень 2022}
\end{quote}

\section*{}\label{section-14}
\addcontentsline{toc}{section}{}

\markright{}

Одиночество подобно покойному родственнику,

Пришедшему навестить тебя на пике отчаянья,

Откровенно осточертело.

Скитания по ветхим перекресткам,

Заносчивых идей и некогда всплывающих из коридоров вечности лиц

Лишено всякого смысла.

И только сумасшествие, укрывшее саваном

Еще вчера исправно тикающий рассудок

Привносит легкое очарование

В бой мыслей лопнувших сосудов.

\begin{quote}
\emph{август 2022}
\end{quote}

\section*{НИГДЕ НЕТ МЕСТА
МНЕ}\label{ux43dux438ux433ux434ux435-ux43dux435ux442-ux43cux435ux441ux442ux430-ux43cux43dux435}
\addcontentsline{toc}{section}{НИГДЕ НЕТ МЕСТА МНЕ}

\markright{НИГДЕ НЕТ МЕСТА МНЕ}

Нигде нет места мне.

В тени земного рая

Иду, стихи слагая

О были давних лет,

Где небеса, сгорая,

Под вечер проливают свет

На краешки могил моих друзей давно ушедших.

И снова шаль отчаянья оттягивает плечи.

Вдыхаешь --- выдыхаешь лишь на миг,

Чтоб вспомнить тех, с кем некогда был счастлив.

Все до того, как мир утих.

И в океане погребенных далей

Гребу один.

Вместо русалок --- продавщицы

С глазами полными вопросов.

Я не боюсь уже допросов:

Не умер --- значит, победил.

Только кого?

Себя. Однако

Уже не выбраться из тени мрака

Тех городов, квартир и зданий,

Где атрибут лежит тоскливый

Моих садов.

Не нужно ксивы, чтоб в них войти:

Замок снесли,

Когда меня туда внесли

Вперед ногами, через гнет

Чужих смятений и хлопот.

И вот лежишь в сырой могиле,

А времена словно застыли.

По ним прошел последний бой

С судьбою или же с собой.

Там небеса в печальной брани

Висят как памятник усталый

И тихо близких голоса

Мне душу жалят как оса.

Закрой, мол, дверь, и уходи.

Но что за нею впереди?

Это не дверь, а крышка гроба,

Где желчи полная утроба

Все не дает покоя мыслям

В мирах, где мы растерянно зависли.

В закате, в браке и во тьме

Уже давно нет места мне.

\begin{quote}
\emph{осень 2022}
\end{quote}

\section*{}\label{section-15}
\addcontentsline{toc}{section}{}

\markright{}

Осень без тебя --- лишь тусклая пора года,

Где искры истлевших воспоминаний

Парят в кавалькаде ушедших событий.

Шахматный клуб --- мое любимое воспоминание.

Во снах мы бродим по нему вместе, а наяву

--- лишь колючая проволока перекрывает спуск к воде.

Мне представляется, как ты пишешь музыку

Один. В углу пианино, под которое нам так не удалось спеться.

Но как же здорово мы слышали этот мир. Вместе.

\begin{quote}
\emph{октябрь 2022}
\end{quote}

\bookmarksetup{startatroot}

\chapter*{2023}\label{section-16}
\addcontentsline{toc}{chapter}{2023}

\markboth{2023}{2023}

\section*{КВАРТАЛЫ
ДУШИ}\label{ux43aux432ux430ux440ux442ux430ux43bux44b-ux434ux443ux448ux438}
\addcontentsline{toc}{section}{КВАРТАЛЫ ДУШИ}

\markright{КВАРТАЛЫ ДУШИ}

От медитации --- к прокрастинации,

Зачистке дум и модерации.

Затем --- к полнейшей трепанации

Своей души.

Ты не спеши.

Пускай еще окурок теплится

Надежд, что обратились вечностью,

Смятений, тягот и погрешностей.

Просто дыши.

Беги туда, где ясно солнышко

И мысли легкие как перышко.

Побудь еще тем хоть мгновение.

Прими покой за сновидение.

Помнишь падение?

Винтовые лестницы с людьми случайными,

Поезд в мимолетные дали с подошвами рваными

И с до изнеможения выгнутыми шпалами.

Расцелуй их губами алыми!

Пройдись напоследок души кварталами

И помаши пешеходам усталым,

Да дай себе обещание:

Рассмеяться им на прощание.

\begin{quote}
\emph{февраль 2023}
\end{quote}

\section*{}\label{section-17}
\addcontentsline{toc}{section}{}

\markright{}

Февраль. Уже поплакал, а что дальше?

Нет толку дни считать в календаре.

Уж лучше бы пройтись по лесной чаще

Да от пуль дыры сосчитать в стекле.

Ведь ими можно рисовать картины,

Не убивать отчаянных людей.

Кварталы стали крайне нелюдимы:

Полны одних аптек да фонарей.

Лить слезы --- бесполезно и банально.

Как только их следы простыли

Глядишь в окно и думаешь сакрально:

Проснуться б в марте мне без этих взрывал.

До крайности обыденные думы,

От их обычности уже тошнит.

Нетривиально что сегодня, скажите?

Похоже, жить!

Да с головой своей дружить.

\begin{quote}
\emph{февраль 2023}
\end{quote}

\section*{}\label{section-18}
\addcontentsline{toc}{section}{}

\markright{}

Огни зажглись. Мосты сгорели.

Когда такое небо над Землей

Войны не может быть,

Но она есть.

И негде в одиночестве присесть,

Чтобы понять:

Мы не вольны, увы, решать,

Какие мысли записать,

А скольким из тех слов истлеть,

Оставшись без издания.

И впредь

Плеть недосказанных мотивов

Будет хлестать рабов строптивых,

Что некогда решились петь.

Что есть мотив, коль не попытка

Души сказать чуть более столь малым

Другим умам --- тоже усталым,

Которым не пришлось сгореть?

Скорей, истлеть,

Чтобы забыть все мимолетные исходы,

Событий, линий переходы

Из жизни в жизнь с одним концом,

И снова ветер под крыльцом

Шуршит, немного утомляя

Покой души,

Что в одиночестве сгорает.

\begin{quote}
\emph{весна 2023}
\end{quote}

\bookmarksetup{startatroot}

\chapter{2024}\label{section-19}

\section{ВИЗИТ}\label{ux432ux438ux437ux438ux442}

Одним дождливым летним днем

Я жизнью обозналась.

В плену неведомых оков

Моя судьба решалась.

Среди аллей цветущих роз,

Под небосводом, полным слез,

Но не моих --- родных, которые смирились,

Что чадо их больно шизофренией.

И мне бы было весело узнать,

Как громко психиатры будут звать

Меня по имени в объятиях сновидений,

Когда на деле смешивают с тенью.

Им не дано понять моих терзаний наяву:

Пускай я пью таблетки и плыву

Куда меня послало наваждение.

Мир для меня --- всего лишь сновидение,

Где не могу проснуться от кошмара

Да заглушить в душе пожара.

Смирюсь иль нет, мы все равно уснем

Одним дождливым летним днем.

\begin{quote}
\emph{август 2024}
\end{quote}

\section{ЛЕС}\label{ux43bux435ux441}

Безоблачный вечер

Иронии вечной

Взгляды, фантазии на века.

Возьму коньяка

Да выпью беспечно,

Слизав лимонного сока

С жизни клинка.

Битые одиноко,

И все же не стану

Я праздности памятник возводить.

Притворно вздохну и перестану

Собою, той, что унылая, быть.

Нитью серебряной воспоминания

Стучат поднебесью в такт.

Одной лишь песнью мир не органичен.

Он вечен и без преград,

Что разум усталый

Рисует вяло

По памяти старой.

Беги без оглядки:

Забвение сладко.

Мотивы малюй,

Скройся в лес

--- там больше небес.

Для них и рисуй.

Укройся под хвои волной,

Стань собой

И в тишине потанцуй.

\begin{quote}
\emph{осень 2024}
\end{quote}

\section{}\label{section-20}

Осень. Открытие. Бар.

Людь сгущаются с сумраком дня.

Скрипки мотив дребезжит словно пар

Мелодий, что звонко звучат без меня.

Сапфиры да вывески --- роскошь праздная,

Когда котелок подгорает.

Музыка вечера --- такая разная,

Если мир без тебя убегает.

Один без любимых. Один в бытии:

Друзья рядом словно картонные.

Сидят, сном забывшись, да не знают они,

Какая ночь бездонная.

И губами прижмусь к вину,

А хотелось бы к микрофону.

Об иной судьбе не молю,

Но и этой не могу быть довольна.

\begin{quote}
\emph{сентябрь 2024}
\end{quote}

\section{}\label{section-21}

Знецінені набутки билини

Мов підсвідомості, що ніччю сяють.

Подумки сплять батьківщини лани

Доки над ними пролітають

Картки примарні із життів минулих,

З дзвінкими посмішками,

Що вже перетнули

Кордони та замкнулись в своїх думах.

І презентований одному лиш собі

Йду коридорами спустошеного міста

Вогні чиїхсь квартир межують вдалині

Мої думки та вогники намиста

Тих фраз,

що я почув колись од вас,

До того як усі немов один

Сховалися за тінню чужини.

Все ж, крила любові колищаться

Сірому небу в такт

Насамоті краще пишеться

Чому й залишаюсь рад.

\begin{quote}
\emph{вересень 2024}
\end{quote}

\section{}\label{section-22}

Я изгнана из своего мира.

Под сенью осеннего дня

Бреду городами сознания,

Сгорая день ото дня.

Бессознательное коллективное

Сжирает мой глас изнутри,

Пока душа строптивая

Не видит путей впереди.

Вновь чудится море и свечи,

Что тускло горят в тишине,

Возлагая на тонкие плечи

Глухую панихиду по мне.

Деревьев листья колышутся,

Затмленному схизой рассудку

Иные миры снова слышатся.

Там никто не подаст руку.

Я стану Буддой в Самбхога-Кайе

Откуда нет возврата,

Но знаю,

Там, где ум Нирване не подвластен,

Мышленье --- бесполезная трата.

\begin{quote}
\emph{осень 2024}
\end{quote}

\section{КОНДИНЦИОНЕР}\label{ux43aux43eux43dux434ux438ux43dux446ux438ux43eux43dux435ux440}

Я хочу выброситься в океан.

Не тот, чем ты был для меня,

А тот, что поглубже

И явно почище.

Я мечтаю изливать мотивы

И опять дарить миру свет,

Но никогда не буду счастлива,

Ведь оставила сердце в шестидесятых.

Я не по тебе сгораю,

А умираю по любви,

Которая во мне была

До того, как мы встретились.

Я могла бы сигануть с обрыва

Элегантно скатиться кубарем вниз

Да раствориться в Бордо мирах,

Но все что мне сейчас интересно:

Льет ли за окном дождь,

Или это опять чей-то кондиционер

Протекает насквозь

Как моя крыша?

\begin{quote}
\emph{сентябрь 2024}
\end{quote}

\section{ПРАХ}\label{ux43fux440ux430ux445}

Струятся отчаянья капли,

Кипит котелок, аж вздымается,

Каркает Ворон под крышей

И лес пустотой обращается.

Рассудок оковами плавится,

Твой голос сердца отбивает такт.

Разве так от чертей избавляются?

Нет --- их изгоняют в Ад.

И на краю забывшись дня,

Предприму последний шаг:

Захочу уйти в себя,

Но нет меня там --- только прах.

Хотела прыгнуть из окна,

Да зацепилась ногой за форточку.

Мир разит как просроченная колбаса

И жить в нем больше не хочется.

Ничто уже не важно под Луной.

Даже души моей покой.

\begin{quote}
\emph{сентябрь 2024}
\end{quote}

\section{МОЙ МИР}\label{ux43cux43eux439-ux43cux438ux440}

Пряди золотых волос

Окаймляют мудрый взгляд,

Что исходит от синих глаз

--- дать совет всегда он рад.

Тихий, спокойный голос

Да такой же краткий нрав.

Справедливость во мне взрастивший

Он всегда был и будет прав.

Обожает с лимоном чай,

Быть в пути и занятия спортом.

Он мне в детстве сказал «Прощай!»,

И мир обратился жестоким.

Номер его телефона

В сердце моем как гравировка.

Я всегда о нем помню,

Как бы ни было плохо.

И с судов возвратившись усталый,

На миг присядет герой, наконец.

Мой мир без него исчезает.

Он --- это мой Отец.

\begin{quote}
\emph{осень 2024}
\end{quote}

\section{НЕСНОСНАЯ
ЕРЕСЬ}\label{ux43dux435ux441ux43dux43eux441ux43dux430ux44f-ux435ux440ux435ux441ux44c}

Изумрудных глаз

Свет погас.

Потускнели мотивы.

Мы были

Чем-то вроде семьи.

Однако ушли

Твои нежные взгляды.

Я рада

Тому, что ты есть.

Не счесть

Тех строк,

Что звучат в моей голове.

Твой голос жесток,

Но только ко мне.

И города серость

--- несносная ересь

Коль в нем без тебя веселиться.

Мне бы не спиться.

Рубаху надену твою

Как в бреду,

Немного повою

И тихо уйду

Туда, где сонеты

Разбитых сердец

Струятся над былью.

Мы были. Нас нет.

Со списка органов мой мозг исключен

Раз помнит твой запах - он обречен.

Мне нравился звон

Звезд, что сошлись

Когда столь негаданно мы разошлись

Вперед без оглядки.

С чего же так сладко

Тебя вспоминать?

В петлю ли опять

Я лезу бездумно

Да в мыслях так шумно.

И я бы сыграла на укулеле

Но знаю лишь один бой:

Других впитать не успела.

Как жаль, что ты не со мной.

\begin{quote}
\emph{сентябрь 2024}
\end{quote}




\end{document}
